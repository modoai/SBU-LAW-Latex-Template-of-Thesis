% در این فایل، عنوان پایان‌نامه، مشخصات خود، متن تقدیمی‌، ستایش، سپاس‌گزاری و چکیده پایان‌نامه را به فارسی، وارد کنید.
% توجه داشته باشید که جدول حاوی مشخصات پایان‌نامه/رساله و همچنین، مشخصات داخل آن، به طور خودکار، درج می‌شود.
%%%%%%%%%%%%%%%%%%%%%%%%%%%%%%%%%%%%
% دانشگاه خود را وارد کنید
\university{شهید بهشتی}
% دانشکده، آموزشکده و یا پژوهشکده  خود را وارد کنید
\faculty{حقوق }
% گروه آموزشی خود را وارد کنید
\degree {کارشناسی ارشد} 
% گروه آموزشی خود را وارد کنید
\subject{دانشکده حقوق}
% گرایش خود را وارد کنید
\field{حقوق محیط زیست}
% عنوان پایان‌نامه را وارد کنید
\title{تاثیر تغییرات اقلیمی بر حق بر آینده سبز}
% نام استاد(ان) راهنما را وارد کنید
\firstsupervisor{دکتر جنت بلیک}
%\secondsupervisor{استاد راهنمای دوم}
% نام استاد(دان) مشاور را وارد کنید. چنانچه استاد مشاور ندارید، دستور پایین را غیرفعال کنید.
\firstadvisor{دکتر باقر انصاری}
%\secondadvisor{استاد مشاور دوم}
% نام پژوهشگر را وارد کنید
\name{محمد مهدی}
% نام خانوادگی پژوهشگر را وارد کنید
\surname{وزیری}
% تاریخ پایان‌نامه را وارد کنید
\thesisdate{1397}
% کلمات کلیدی پایان‌نامه را وارد کنید
\keywords{تغییرات اقلیمی، آینده سبز، حقوق کودک، حقوق نسل های آینده}
% چکیده پایان‌نامه را وارد کنید
\fa-abstract{\noindent
نسل‌های گذشته و حاضر با دامن زدن به تغییرات‌اقلیمی، نظم و الگوی دما و بارش را در سطح کره زمین تغییر داده‌اند، به نحوی که پایداری و ثبات اقلیم دچار اختلال شده است. بازگشت اقلیم به وضعیت ثبات چیزی حدود ده هزار سال زمان نیاز خواهد داشت و این به معنای وجود چالش در رابطه بین نسل‌های حاضر و آینده است. 
در این میان، باید پرسید که تغییرات‌اقلیمی چه تاثیری بر ذینفعان حق بر آینده سبز می‌گذارد. همچنین تعهد نسل‌های حاضر در برابر حق بر آینده سبز در ساختار حقوق بین‌الملل محیط‌زیست چیست؟
%تعهدات مختلفی در کنوانسیون‌های حقوق بین‌الملل محیط‌زیست برای مقابله با این پدیده تدارک دیده شده است، ماهیت نرم این تعهدات و منفعت‌طلبی دولت‌ها با تکیه بر اصل حاکمیت دولت‌ها، مانع شکل‌گیری اجماع بین‌المللی موثر در برابر این پدیده و حمایت از حق بر آینده سبز بوده است. \\
تغییر دما و بارش، با ایجاد محدودیت در دسترسی به منابع، کودکان را از نیاز‌های اساسی، حقوق خاص، حقوق مشارکتی  و زندگی در جامعه‌ای پایدار که حقوق اقتصادی، اجتماعی، فرهنگی، سیاسی و مدنی آنها تضمین شده باشد، محروم می‌نماید. کودکان و نسل‌های آینده، شهروندان آینده زمین هستند که حق آنها بر دریافت زمین به مثابه میراث مشترک بشریت در کمال صحت از نسل‌های حاضر، در حال نقض است. تغییرات‌اقلیمی با محدود نمودن منابع و تخریب آنها، آزادی انتخاب کودکان و نسل‌های آینده را تحدید می‌نماید.\\
جامعه بین‌المللی، در این حوزه با تاکید بر خیر مشترک، عدالت و جامعه ارزش‌های بنیادین بین‌المللی و با تکیه بر اصول همکاری، مسئولیت مشترک اما متفاوت و... توانسته گام‌های در مقابله با تغییرات‌اقلیمی در قالب کنوانسیون چارچوبی تغییرات‌اقلیمی کنوانسیون تنوع زیستی و دیگر اسناد بردارد. اما الزام‌آور نبودن این تعهدات در کنار رقابت اقتصادی دولت‌ها و اصل حاکمیت دولت‌ها در ساختار‌های قدیمی حقوق بین‌الملل از نتیجه بخش بودن این اقدامات کاسته، به نحوی که انتشار گاز‌های گلخانه‌ای روند افزایشی داشته و اجماع بین‌المللی در این حوزه با موفقیت کامل روبه‌رو نبوده است و آینده‌ای سبز را نوید نمی‌دهد. 
در فرایند نگارش سعی گردیده با رویکرد آینده‌نگرانه به موضوع نگریسته شود و از رهیافت ایده‌آلیستی در تببین آرمان‌های جامعه بین‌المللی و نقش آنها در تنظیم رابطه نسل حاضر با نسل‌های آینده استفاده شده است. }
\newpage
\thispagestyle{empty}
\vtitle
\newpage
\thispagestyle{empty}
\clearpage
~~~
\newpage
\thispagestyle{empty}
\input{rights}
\newpage
\thispagestyle{empty}
\clearpage
~~~
%\newpage
%\thispagestyle{empty}
%\centerline{{\includegraphics[width=20 cm]{replyrecord}}}
%\newpage
%\thispagestyle{empty}
%\clearpage
%~~~
\newpage
 % پایان‌نامه خود را تقدیم کنید!
\begin{acknowledgementpage}

\vspace{4cm}

{\nastaliq
{\Large
تقدیم به مادرم که بذرهای امیدواری می‌کارد، تا من درو کنم.

و تقدیم به عبدالحسین وهاب‌زاده، موسس مدارس طبیعت که دغدغه طبیعت و نسل‌های آینده را دارد.
\vspace{1.5cm}

\newdimen\xa
\xa=\textwidth
\advance \xa by -11cm
\hspace{\xa}
با مهر
}}
\end{acknowledgementpage}
\newpage
\thispagestyle{empty}
\clearpage
~~~
%%%%%%%%%%%%%%%%%%%%%%%%%%%%%%%%%%%%
%\newpage
\thispagestyle{empty}
% ستایش
\baselineskip=.750cm
\ \\ \\
 
{\nastaliq
%رسيدن، به دانش است و به كردار نیک...%
}\\
\vspace{.5cm}\\
{\scriptsize\nastaliq
{



«وَمِنَ النَّاسِ مَنْ یُعْجِبُکَ قَوْلُهُ فِی الْحَیَاةِ الدُّنْیَا وَیُشْهِدُ اللَّهَ عَلَىٰ مَا فِی قَلْبِهِ وَهُوَ أَلَدُّ الْخِصَامِ،(204) وَإِذَا تَوَلَّىٰ سَعَىٰ فِی الْأَرْضِ لِیُفْسِدَ فِیهَا وَ یُهْلِکَ الْحَرْثَ وَالنَّسْلَ  وَاللَّهُ لَا یُحِبُّ الْفَسَادَ (205)» 

و پاره اى از مردم ، منافق و سالوسند كه وقتى سخن از دين و صلاح و اصلاح مى كنند تو را به شگفت مى آورند و خدا را گواه مى گيرند كه آنچه مى گويند مطابق آن چيزى است كه دردل دارند و حال آنكه سرسخت ترين دشمنان دين و حقند. (204)
(به شهادت اينكه ) وقتى بر مى گردند (و يا وقتى به ولايت و رياستى مى‌رسند) با تمام نيرو در گستردن فساد در زمين مى كوشند، كه او از راه نابود كردن حرث و نسل در زمين فساد مى انگيزد،و در نابودى انسان مى‌كوشد.(205)\footnote{قرآن کریم، سوره بقره، آیات ۲۰۴ و ۲۰۵.}

.

.

.

.

.

.

.
«زیرامن که یهوه، خدای تو می‌باشم، خدای غیورهستم، که انتقام گناه پدران را از پسران تا پشت سوم و چهارم از آنانی که مرا دشمن دارند می‌گیرم. و تا هزار پشت بر آنانی که مرا دوست دارند و احکام مرا نگاه دارند، رحمت می‌کنم.»\footnote{کتاب مقدس، عهد عتیق، سفر خروج، باب بیستم،(ده فرمان)، آیات پنجم و ششم}
.

.

.

.

.

.

.

.

صحبت از آینده بدون امید بی‌هوده خواهد بود و عبث

امید‌های سبزمان را به آینده‌ای آزاد برای حافظان اقلیم ایران گره می‌زنیم، 

برای

امیرحسین خالقی، هومن جوکار، طاهر قدیریان، سام رجبی، نیلوفر بیانی، سپیده کاشانی، مراد طاهباز، و عبدالرضا کوهپایه
 }}
 
\vspace{.5cm}
{\nastaliq
\newdimen\xb
\xb=\textwidth
\advance \xb by -8.5cm
\hspace{\xb}
%پس منطق ناگزير آمد بر جوينده‌ی رستگاری.

}
\newpage
\thispagestyle{empty}
\clearpage
~~~
%%%%%%%%%%%%%%%%%%%%%%%%%%%%%%%%%%%%
\newpage
\thispagestyle{empty}
% سپاس‌گزاری
{\nastaliq
%سپاس‌گزاری...
}
\\[2cm]


به پاس شیرینی دقایقی که خود را غرق  تعمق در مفاهیم بشریت، عدالت و انصاف و رابطه آن با کودکان و نسل‌های آینده می‌یافتم، از اساتید گروه حقوق محیط‌زیست، خصوصا خانم دکتر بلیک و آقای دکتر انصاری که زحمت راهنمایی و مشاوره این اثر را بر عهده داشتند، تشکر نموده و خود را مدیون آنان می‌دانم. 

همچنین از آقای دکتر محسن عبدالهی، که داوری این اثر را بر عهده گرفتند‍، بواسطه نقد‌هایی که چراغ راه پژوهش‌های آتی من است تقدیر و تشکر می‌نمایم. 

% با استفاده از دستور زیر، امضای شما، به طور خودکار، درج می‌شود
\signature 
\newpage
\thispagestyle{empty}
\clearpage
~~~
\newpage
%{\small
\abstractview
%}
\newpage
\thispagestyle{empty}
\clearpage
~~~
\newpage