\clearpage
\phantomsection
\addcontentsline{toc}{chapter}{نتیجه‌گیری}
\chapter*{نتیجه‌گیری}\markboth{نتیجه‌گیری}{نتیجه‌گیری}



%برآیندی از تحقیق
در این نوشته کوشیدیم تا مسئله تغییرات‌اقلیمی را بر حق بر آینده سبز در دو فصل بررسی کنیم. از یک سو با تامل در گزارش‌های هیات بین دولتی تغییرات‌اقلیمی، آثار این پدیده را بر زمین، حقوق کودکان و نسل‌های آینده مطالعه کردیم و از سوی دیگر رویکرد جامعه بین‌المللی را مد نظر داشتیم و در کنار آن تحولات در عرصه مفاهیم و اصول حقوقی را تبیین نمودیم. در طول مسیر بر آن بودیم که تا حد ممکن تاثییرات این جریان‌ها بر یکدیگر و آینده‌ای که شکل می‌دهند را تا حد ممکن روشن و واضح بیان نماییم.

در فصل اول، ابتدا به تبیین دورِ انسان پرداختیم. و دریافتیم که این چرخش در ادوار زمین‌شناختی، دارای آثاری همچون بی‌ثباتی و تخریب الگوی پایداری است. تغییرات‌اقلیمی به عنوان یکی از نشانه‌های عدم پایداری  به ما نشان داد که با پدیده‌ای روبه‌رو هستیم که برای حداقل چند صد سال تمام وجوه زندگی موجودات را در زمین درگیر خود خواهد نمود. این چالش به درستی رابطه نسل‌‌های حاضر با نسل‌های آینده را که به صورت پیش‌فرض  می‌بایست مبتنی بر عدالت و انصاف و احترام باشد را، تخریب نموده و چالش‌های عدیده‌ای در مسیر دستیابی نسل‌های آینده به حق بر آینده سبز ایجاد نموده است. 

کودکان و نسل‌های آینده، نسبت به محیط‌زیست پاک و سالم محق هستند. آنها دارای حق بر آینده سبز، روشن و همراه با امیدواری نسبت به آینده می‌باشند. در فصل دوم، طی دو مبحث به بررسی حقوق کودکان و نسل‌های آینده پرداختیم. در مبحث اول، ضمن تبیین حقوق کودک، موارد نقض این حقوق توسط تغییرات‌اقلیمی، طی پنج گفتار تشریح گردید. مبحث دوم با ماهیت و چالش‌های نسل‌های آینده آغاز گردید. نسل‌های آینده خلاف کودکان، به دلیل موجود نبودن، در حوزه حقوق مورد غفلت قرار می‌گیرند. این چالش به همراه آگاهی از خطرات تغییرات‌اقلیمی برای آنها، دشواری حمایت از این گروه را در عالم حقوق نمایان می‌سازد. در این مبحث، کوشیدیم ضمن توجه به نظریه سود یا منفعت، مبنایی برای حقوق آنها بیابیم، سپس در عالم واقع به جستجوی منابعی پرداختیم که نسل‌های آینده در آن مورد توجه بوده است. و در نهایت، رابطه تغییرات اقلیمی و نسل‌های آینده مورد اشاره واقع شد. 

تا اینجای کار، به معرفی و چالش‌های تغییرات‌اقلیمی بر حق بر آینده سبز پرداختیم. در فصل پایانی، پاسخ‌های جامعه بین‌المللی و مبنای این پاسخ‌ها را مد نظر داشتیم. این فصل ابتدا با معرفی هیات بین‌دولتی تغییرات‌اقلیمی به عنوان یک نهاد علمی در جهت‌دهی به افکار عمومی و سیاسی دولت‌مردان آغاز شد. جامعیت گزارش‌های این نهاد در فهم جدی بودن تهدید تغییرات‌اقلیمی نقشی بسیار اساسی داشته و سندی مطمئن جهت تصمیم‌گیری در کنفرانس‌های اعضا و دیگر نهاد‌ها می‌باشد. 

 کنوانسیون چارچوبی تغییرات‌اقلیمی و اسناد مصوب کنفرانس اعضا که شامل پروتکل‌ کیوتو و موافقت‌نامه پاریس هستند به عنوان اسناد تخصصی در حوزه مقابله با تغییرات‌اقلیم مورد بررسی قرار گرفت. 
 در گفتار مربوطه به تشریح ساختار و ارکان کنواسیون، تعهدات دولت‌های عضو و قواعد شکلی کنوانسیون در کنوانسیون مادر و پروتکل‌های آن پرداختیم. 
 در دو گفتار پایانی مبحث، به کنوانسیون تنوع زیستی و جنگل‌ها بواسطه همبستگی طبیعی اقلیم و تنوع‌زیستی  و نقش جنگل‌ها در تثبیت دما و بارش پرداخته و رابطه نزدیک میان این دو را تشریح نمودیم.
 
 در نهایت در مبحث پایانی فصل سوم، آرمان‌های حمایت از حق بر آینده سبز در برابر تغییرات‌اقلیمی مورد توجه واقع شد. در خصوص مفهوم بشریت، به میراث مشترک بشریت، نگرانی مشترک بشریت و خیر مشترک بشریت پرداخته و سعی نمودیم روند پیدایش و جایگاه آنها را تبیین نماییم. همچنین در بحث از عدالت و انصاف، به نحوه ارتباط این مفهوم با اصول تغییرات‌اقلیمی و جایگاه آن پرداختیم. 
 
 
 
 
 
 
 
 %پاسخ به سوالات تحقیق
 
 
 
 
 تغییرات‌اقلیمی با ظهور انقلاب صنعتی و تغییر دورِ هالوسن به دورِ انسان آغاز می‌گردد. این تغییرات پس از انباشت گسترده گاز‌های گلخانه در جو بآهستگی در سطح وسیعی نظم و الگوی بارش و دما را تغییر می‌دهد. این در حالی است که جامعه بین‌المللی در سال 1972 با انتشار اعلامیه استکهلم توجه عموم کشور‌ها را به مسئله محیط‌زیست جلب می‌نماید و نهایتا در سال 1992 در نشست ریو اسناد الزام‌آور در حوزه حفاظت از تنوع زیستی و تغییرات‌اقلیمی به تصویب می‌رسند. 
 
 آغاز دورِ انسان به معنای ایجاد تغییرات وسیع به لحاظ مکانی و زمانی است. اقلیم ناپایدار تا چندین عمر انسانی را تحت تاثیر خود قرار می‌دهد. این امر از طرفی با تخریب منابع، نیاز‌های اساسی کودکان برای رشد سالم را نابود می‌سازد، از طرف دیگر، وضعیت آینده جوامع را با شهروندانی که در محیط‌زیست تخریب شده رشد یافته‌اند، تهدید می‌نماید. کودکان، در اثر حوادث ناشی از تغییر اقلیم، از حق بر آموزش، تفریح، تابعیت، حق بر فرهنگ و رشد در جامعه پایدار محروم خواهند شد. همچنین این تغییرات با به زیر آب بردن بعضی کشور‌ها و دیگر رویداد‌ها، حقوق و منافع نسل‌های آینده را بر رشد و زندگی در جامعه پایدار اقتصادی، اجتماعی و محیط‌زیستی به صورت جدی تهدید می‌نمایند. 
 
 
 
 
 
 
 
 
 
 ما نسل ناظر دگرگونی زمانه هستیم. یعنی اندیشیدن به آینده و رویداد‌های محتمل آن بیش از گذشته و دگرگونی‌های آن باید در فرایند‌های تصمیم‌گیره و قاعده‌گذاری مد نظر باشد. 
 
 در حال حاضر بیش از آنکه شاهد زمانه دگرگونی‌ها باشیم، ناظر دگرگونی زمانه هستیم. یعنی این تغییرات زمانه است که بر ساختار روابط اجتماعی تاثیر می‌گذارد و ضرورت همگام‌سازی قواعد اجتماعی را نمایان می‌سازد. 
 
 در حقوق بین‌الملل سنتی که با ویژگی‌هایی همچون اصل حاکمیت، خود‌محور بودن یا خصوصی بودن روابط میان دولت‌ها و اصل اراده خود‌پسندانه منافع دولتی مورد شناسایی قرار می‌گیرد، به دلیل عدم وجود حاکمیت مرکزی و آنارشی در روابط تابعان آن، منابع حقوق عمدتا در پی حل تعارضات گذشته شکل گرفته و بوجود می‌آیند. این امر در حقوق داخلی معکوس است. بدین معنا که قانون‌گذار با ذهنیتی آینده‌نگرانه به وضع قاعده می‌پردازد.
 
 
 حقوق تغییرات اقلیمی با توجه به آثار بلند مدتی که تغییرات اقلیمی دارد در پی آن است که با ابزار‌های موجود حقوق، نظم کنونی جامعه و طبیعت را به آینده تسری داده و وضعیت حاضر را تا آینده استمرار بخشد. 
 
 برای این کار حقوق محیط‌زیست ضمن حمله به اصل اساسی نظم بین‌الملل سنتی مانند اصل حاکمیت دولت و فرسایش منفت‌گرایی ملی با تمرکز بر اصول توسعه پایدار، اصل همکاری، مسئولیت مشترک اما متفاوت و ... که ریشه در ارزش‌های بنیادین جامعه بین‌المللی همجون بشریت، عدالت و انصاف بین‌نسلی دارند، در قالب حقوق بین‌الملل همزیستی به حقوق آیندگان از جمله حقوق کودکان و نسل‌های آینده توجه می‌نماید. 
 
 زمان با به تحرک در آوردن و تغییر یافتن، ارزش‌های حاکم را از میان می‌برد و فضا را برای ارزش‌های جدید ایجاد می‌نماید. 
 
 
 
کنوانسیون تغییرات‌اقلیمی با طراحی مکانیسم‌های مختلف در جهت نیل به خیر مشترک و عدالت توزیعی قدم بر داشته و اجماع خوبی در سطح جهانی برای مقابل با تغییرات‌اقلیمی شکل داده است. این عدالت با توجه به چالش نابرابری مسئولیت کشور‌ها و همچنین نابرابری آثار تغییرات‌اقلیمی بر افراد انسانی مفاهیمی همچون مسئولیت مشترک اما متفاوت و توسعه پایدار را ایجاد نموده است. با این حال، می‌بینیم که تاخیر در آغاز مقابله با تغییرات‌اقلیمی نمایانگر ناتوانی سازکار‌های جامعه بین‌المللی در شناخت و مقابله با تهدیدات آن است. در واقع به نظر می‌رسد در حوزه حقوق محیط‌زیست ما با تابعان اجتماعی بین‌المللی به جای جامعه‌ای با اهداف مشخص و معین روبه‌رو هستیم. در این حالت، اولویت داشتن منافع ملی بر منافع بین‌المللی از جمله عوامل بی توجهی به وضعیت میراث مشترک بشریت در کل است. 
 
در مبحث اول از فصل سوم، دیدیم که جامعه بین‌المللی در مقابله با تغییرات‌اقلیمی در کنار انتشار روزافزون گاز‌های‌ گلخانه‌ای موفقیت چندانی بدست نیاورده است. اسناد بین‌المللی اگرچه توجه خود را به آینده معطوف داشته‌اند اما کماکان در تبیین حقوق نسل‌های آینده و ایجاد تعهدات جدی در این حوزه ناکام مانده‌اند. ناتوانی جامعه بین‌المللی در ایجاد همبستگی جهانی و عدم وجود اراده سیاسی ارزش‌محور در این حوزه از جمله موانع اساسی در مسیر دستیابی به حق بر محیط‌زیست سالم در آینده خواهد بود. 
 
 حقوق اگر چه توانسته در قالبی قراردادی کشور‌های توسعه‌یافته و در حال توسعه را به پای میز مذاکره نشانده و در قالب حقوق، قدرت آنها را در انتشار گاز‌های گلخانه‌ای محدود نماید، اما پر واضح است که نتوانسته در این حوزه بر قدرت حاکمیت دولت‌ها با ابزار ارزش‌های بین‌المللی نیل آید. چنین ساختاری اگرچه منتج به نتیجه دلخواه در عرصه مقابله با تغییرات‌اقلیمی می‌گردد، اما در ساختار‌های منفعت‌گرایی دولت‌ها و قدرت نامنتهای آنها جای گرفته است. همین امر، موجب می‌شود تا در صورتی که دولتی قواعد قراردادی این حوزه را خلاف منافع ملی خود شناسایی کند از پذیرش تعهدات در این حوزه شانه خالی نموده و منفعت ملی خود را به خیر مشترک جامعه جهانی ترجیح دهد. 
 
 به عبارت دیگر، ارزش‌های بنیادین جامعه بین‌المللی در برابر ملی‌گرایی کشور‌ها و اصل حاکمیت وستفالیایی دارای قدرت اجرایی نبوده و جامعه بین‌المللی در مرحله شناسایی و تفهیم این مفاهیم قرار دارد. چنین موضوعی، به همراه آمارها و ارقام نگران کننده در حوزه افزایش انتشار گاز‌های گلخانه‌ای طی سال‌های گذشته بیانگر ناتوانی سازکار‌های بین‌المللی در حوزه مقابله با تغییرات‌اقلیمی و حمایت از حق بر آینده سبز می‌باشد.
 
 
 
 %تعیین سرنوشت فرضیه
 
در ابتدا نگارنده تاثیرپذیری آیندگان از تغییرات‌اقلیمی به دلیل ناتوانی جامعه جهانی در مقابله با این تهدید را به عنوان  فرضیه اصلی این متن مطرح نمود. در پایان مشاهده می‌نماییم که اقلیم، این میراث مشترک بشریت که تبدیل به نگرانی مشترک بشریت گردیده در ناکارآمدی ساختار‌های بین‌دولتی در حال تخریب بوده و اقدام مثبتی تا کنون در جهت مقابله با این پدیده برداشته نشده است. بنابراین صحت فرضیه اصلی تایید گردیده و فرضیه رقیب رد می‌شود. 
 
 %یافته های فراتحقیقی
 
متن حاضر با چند پرسش مختصر آغاز شد و اکنون که این اثر به پایان رسیده، دستاورد‌های بسیاری برای نگارنده به همراه داشته است. از جمله این موارد توجه به مبنای سرمایه‌داری و دولت‌های مبتنی بر سرمایه‌داری در ایجاد چالش‌های  محیط‌زیستی و آغاز دورِ انسان می‌باشد. همچنین توجه به ژن خودخواه و روند تکامل انسان در طبیعت تا به امروز که پاسخ بسیاری از مشکلات در آن وجود دارد از جمله یافته‌های فراتحقیقی می‌باشد که در متن حاضر بدان اشاره نگردید.  

%پیشنهادات و راهکارها

در حوزه پیشنهادات و راهکار‌ها، نگارنده نقطه ثقل توجهات را از دولت‌ها برداشته و با توجه به ناکارآمدی آنها، پیشنهاد می‌نماید بر افراد و اصلاح الگوی مصرف، از طرق مختلف خصوصا در عصر فناوری کنونی و با توجه به در دسترس بودن رسانه‌های جهانی تاکید شود. محور قرار دادن بشر به صورت مستقیم و در نظر گرفتن دولت‌ها به عنوان ساختار‌های تسهیل کننده شرایط زندگی برای مردم به رفع موانع حقوق بین‌الملل و ایجاد عدالت مبادله‌ای میان شهروندان کشور‌های مختلف با تابعیت‌های متفاوت کمک خواهد نمود. اگر تفاوتی میان تابعیت شهروندان افغانستان و انگلیس نباشد، ملی‌گرایی رنگ باخته و فرصت برای نیل به خیر مشترک جامعه جهانی باز خواهد شد. 

تاکید جامعه جهانی بر اجتماعات، به موازات دولت‌ها و شنیدن و رسا نمودن صدای اجتماعات و گروه‌های اجتماعی نقشی اساسی در ایجاد فشار سیاسی بر دولت‌مردان از درون دولت‌ها خواهد داشت. 
در این حوزه نسل‌های جوان و سازمان‌های مردم نهاد از جمله پتانسیل‌های تاثیر‌گذار در جهت دهی افکار عمومی جوامع خواهند بود که با بالابردن هزینه‌های روانی برای سیاست‌مداران انتخاب‌هایی را که عامل انتشار گاز‌های گلخانه‌ای است را مورد توجه قرار داده و محدود می‌نمایند. 

همچنین در حوزه آموزش، برای رسیدن به خیر مشترک جهانی و جلوگیری از بروز و رشد نگرانی مشترک بشریت در جامعه جهانی، زمین نیازمند شهروندانی جهانیست. افزایش آگاهی و دانش افراد بشری نسبت به قراردادی بودن مرز‌های سیاسی و وجود منافع مشترک جهانی به صورت گسترده می‌تواند از طرفی عامل مقابله با سیاست‌های مخرب باشد و از طرف دیگر با ممانعت در فرهنگ مصرف گرایی و ایجاد بستر‌های لازم برای ترویج فرهنگ قناعت میان عموم از فشار موجود بر تولید بکاهد و انتشار گاز‌های گلخانه‌ای را کاهش دهد. 


تکالیف نسل حاضر در برابر نسل‌‌های آینده این است که ابتدا با چشمانی باز به آثار عملی که در زمان حال انجام می‌دهد در بازه زمانی آینده و زمانی که نیستند توجه نماید و با رویکرد محتاطانه اقدام به انجام اعمالی نمایند که در تعارض با حقوق بنیادین بشر (چه نسل حاضر و چه آینده) نباشد. از طرف دیگر نهاد‌‌های اجتماعی همچون دولت‌ها و سازمان‌ها، به نمایندگی از بشریت به عنوان تسهیلگر این حقوق مسیر‌‌های نقض آن را سد نموده و مانع شوند.

آموزش و فرهنگ‌سازی مسیر خروج از خودخواهی است. نسل دیگر خواه در مسیر تصمیماتش به شهروندان دیگر کشور‌ها نیز توجه می‌نماید. همچنین در صورت عملی ساختن نظریه عدالت جان رالز، و قرار دادن تصمیم‌گیران (چه خرد و چه کلان) در پس پرده جهل می‌توان به ایجاد بستر‌های مناسب برای نیل به آینده سبز امید داشت. 


%خاتمه نتیجه گیری با طرح سوال
در نهایت این پرسش باقیست که آیا تهدیدات محیط‌زیستی خواهند توانست ساختار‌های قدیمی حقوق بین‌الملل را به تحرک درآورده و با ظهور قواعد آمره در این حوزه اصل حاکمیت دولت‌ها را به زیر بکشند؟ چه بازه زمانی نیاز است تا از تصمیم‌گیری‌های کوتاه‌مدت دولت‌ها عبور کرده و بتوان در سایه حکومت ارزش‌های بنیادین جهانی و بشریت، رفتار تابعان آگاه زمین را منطبق با شرایط و نیاز‌های موجود تبیین نمود؟ 
 

هزاران نقطه روشن بر پیکره آسمان شب مشعل امید را در نظر هوشیار انسان خردمند گرم‌نگه می‌دارند. ستارگانی که نورشان از چند صد سال‌ نوری می‌گذرد و در حالی به ما می‌رسد که ممکن است خودشان چندین سال پیش مرده باشند. نور‌هایی از گذشته‌های دور، در دل تاریکی و ظلمت بی‌انتهای شب بر حال و آینده ما می‌تابند. در وجود برای عدم می‌سوزند و در عدم، وجود را زیبا می‌سازند. آیا از وجود امروز ما برای نسل‌های آینده نوری خواهد تابید؟ 


 
 
 %در وضع کنونی حقوق بین‌الملل محیط‌زیست که در وضع طبیعی بسر برده و توافق بر سر مقابله با تهدیدات اساسی که صلح و امنیت کنونی تابعان آن را هدف گرفته است، به دشواری حاصل می‌شود، آنهم توافقی نه چندان الزام‌آور، 
 
 
 
 
 
 







