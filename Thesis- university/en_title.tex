% در این فایل، عنوان پایان‌نامه، مشخصات خود و چکیده پایان‌نامه را به انگلیسی، وارد کنید.
% توجه داشته باشید که جدول حاوی مشخصات پایان‌نامه/رساله، به طور خودکار، رسم می‌شود.
%%%%%%%%%%%%%%%%%%%%%%%%%%%%%%%%%%%%
\baselineskip=.6cm
\begin{latin}
\latinuniversity{Shahid Beheshti University}
\latinfaculty{Faculty of Law}
\latindegree{Master of Laws (LLM) }
%group:
\latinsubject{Department of Environmental Law}
\latinfield{ Law}
\latintitle{Impact of Climate Change on the Right to a Green Future}
\firstlatinsupervisor{Dr. Janet Blake}
%\secondlatinsupervisor{Second Supervisor}
\firstlatinadvisor{Dr. Bagher Ansari}
%\secondlatinadvisor{Second Advisor}
\latinname{Mohammad Mahdi}
\latinsurname{Vaziri}
\latinthesisdate{2019}
\latinkeywords{Climate Change, Green Future, Child Rights, Future Generation }
\en-abstract{\noindent
The anthropogenic climate change caused by previous and present generations is the biggest global threat of the 21st century. we are able now to shape the future by intervening in the household of nature to a great extent, for millions of years. 
The purpose of this thesis is to show how climate change and
the associated phenomena harm the right to a green future which is beneficial for children and future generations.
Various questions concerning the impact of climate change on the right to a green future, and how the present generation is responsible for future generations, raised.\\
Although International environmental law community responses this treats by shaping different obligations, the right to a green future have Have not received enough legal protection. Some of the obstacles is about International law structures which is based on Principle of sovereignty and also the soft nature of environmental law obligationas.
international community introduced legal concepts such as common concern of humankind, common heritage of humankind and common good of humankind that imply the preservation of environment for future generations. But to be realistic international efforts was not enough due to scientific reports of Intergovernmental Panel on Climate Change(IPCC). 
}
\latinvtitle
\end{latin}