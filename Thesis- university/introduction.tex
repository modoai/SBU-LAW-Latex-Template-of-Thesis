\clearpage
\phantomsection
\addcontentsline{toc}{chapter}{مقدمه}
\chapter*{مقدمه}\markboth{مقدمه}{مقدمه}


\begin{enumerate}
	
	\item پیش درآمد تحلیلی
	
	%جایگاه حق محیطزیست و انسان
	طبیعت طی هزاران سال گذشته در روند تکامل و شکوفایی قرار داشته است. این محیط‌زیست با ایجاد نظم و قواعدی دقیق برای زندگی هرگونه، به شیوه‌ای بسیار خارق‌العاده توانسته مرز‌ها و حریم زیستی هر گونه را با توجه به  نیاز‌های او تنظیم نماید. هوشمندی طبیعت در ایحاد ساختار‌های ژنی است که بواسطه آن هیچ‌ گونه‌ای خارج از حریم نیاز‌هایش اقدام نمی‌کند، پس فرصت برای شکوفایی باقی می‌ماند و حیات همراه با تنوع‌زیستی فراهم می‌شود. 
	
طبیعت، برای انسان  به عنوان یکی از فرزندانش قواعدی حداقلی تعریف نموده که پاسدار زیست او در زمین است. البته طبیعت با در اختیار قرار دادن هوشمندی در انسان به او قدرت اختیار نیز داده است. صورت این اختیار در تنظیم قواعد زندگی در حقوق جلوه‌گر می‌شود. حقوق محیط‌زیست، قواعد و راهکار‌هایی است که با تحدید  آزادی انسان، آرمان عدالت زیستی را به مثابه قاعده و قانون زندگی تعیین می‌نماید. 

وظیفه متصدی حقوق، تبیین رابطه میان دو طرف یک قضیه و استخراج حکم، با توسل بر آرمان، اصل و قاعده حقوقی است. گاهی رابطه پیرامون انسان با انسان دیگر است. همچنین گاهی چنین رابطه‌ای میان انسان و محیط پیرامون او شکل می‌گیرد. و البته ممکن است این رابطه، میان وجود و عدم رخ دهد. 
	
	%نظام بین المللی و ویژگی های آن
	ساختار حقوق بین‌الملل مبتنی بر نظام آنارشی و عدم وجود حاکمیت و قدرت واحد سیاسی است. دولت‌، همزمان که بزرگترین تابع حقوق بین‌الملل است، خود متبوع این نظام بوده و با قواعدی که خود تعیین می‌کند، متعهد می‌شود. در این نظام، اصولا قاعده در پی پاسخ و احقاق نظم از میان رفته گذشته است. یعنی ابتدا چالش‌ و مسئله بین‌المللی میان تابعان رخ می‌دهد سپس در پاسخ به این چالش‌ها نظام حقوقی به وضع قواعد جدید می‌پردازد. 
	
	%زمان طبیعت و زمان انسان
	زمان در روند شکل‌گیری و پیدایش طبیعت و محیط‌زیست از الگویی برابر با زندگی انسان بهره نمی‌برد. در روند پیدایش یک گونه، چرخه‌های بیشمار طی قرن‌ها بر یکدیگر اثر گذاشته‌اند تا آن گونه در قالب امروزین خویش متجلی شود. بنابریان ایجاد تغییرات‌ در طبیعت و محیط‌زیست را نمی‌توان در بازه‌های زمانی کوتاه مدت و تک بعدی نگریست. 
	
	% تغییر وزن مداخله انسان و طبیعت در یکدیگر
	رابطه انسان با طبیعت در کلیت آن در گذشته، بواسطه قدرت و تفوق طبیعت، قهری و یک‌جانبه بود. امروزه این رابطه، پس از رشد روز‌افزون جمعیت، پزشکی و فناوری، که عمدتا پس از انقلاب صنعتی رخ داد، به نحوی تغییر کرده که دیگر شناخت زمین بدون در نظر گرفتن آثار و رد پای انسان غیرممکن است. دامنه آثار انسان بر پیکره سیاره حیات چنان گسترده است که زمین‌شناسان این عصر را دور انسان نامیده‌اند. 
	
	خلاف دور زمین‌شناختی هالوسن که با ویژگی ثبات در شرایط طبیعی و محیط‌زیست شناسایی شده است، ویژگی عمده دورِ‌انسان،  تغییر، عدم‌اطمینان و بی‌ثباتی در رفتار زمین است. این امر به معنای تغییرات گسترده در چرخه‌های زیستی و به تبع آن در روابط اجتماعی و ساختار‌هایی است که مبتنی بر ثبات دور هالوسن شکل گرفته‌اند.\LTRfootnote{Vidas, Davor, «The Earth in the Anthropocene – and the World in the Holocene?», ESIL Reflections, Vol. 4, No. 6, August 2015, P. 2. }
	
	
«با به خطر افتادن محیط زیست انسان اساسی ترین حق انسان یعنی حق حیات وی به خطر می‌افتد. به همین خاطر هر دولتی باید حیات اجتماعی را چنان سامان دهد تا تضمینی برای حفظ مبانی و پایه‌های طبیعی حیات باشد. از این رو مسئولیت در قبال نسل آینده در وهله اول متوجه دولت است و آن را ملزم می‌دارد تا با اتخاذ تدابیری در سطح ملی و بین‌المللی این مسئولیت را به انجام رساند.»\LTRfootnote{Gündling, Lothar, «Our Responsibility to Future Generations», The American Journal of International Law, Vol. 84, No. 1, 1990, P. 202.}
	
	از جمله تغییراتی که در دورِانسان رخ داده، تغییرات‌اقلیمی است که دامنه آثار آن تا سال‌های متمادی ادامه خواهد داشت.
	چنین پدیده‌ای مسئله‌ و چالش حقوق آیندگان را پدید می‌آورد. به این معنا که با تخریب ساختار‌های پایه‌ای که زندگی و تمدن بر پایه آن شکل گرفته برای سال‌های متمادی، نه تنها حقوق نسل حاضر، که زندگی، منافع و حقوق نسل‌های آینده را بشدت متاثر ساخته و دگرگون می‌نماید. بنابر چنین دگرگونی در رابطه انسان و طبیعت با آینده، ضرورت ایجاد «قاعده» حقوقی در پاسخ به «مورد» (چالش) برای متصدی حقوق تبیین می‌گردد.
	
	
	«هیچ‌گاه نمی‌توان نظامِ حقوقی حاکم را فارغ از حقوق دوران گذشته درک کرد، و حقوق آینده را بدون تسلط بر منابع موجود و پیش بینی نتایج آتی بسط داد، زیرا قاعده حقوقی باید هم بر گذشته حکومت کند و هم زمان حاضر را به نظم درآورد و هم برای آینده تمهیداتی بیندیشد.»\footnote{فلسفی، هدایت‌الله، \textbf{سیر عقل در منظومه حقوق بین الملل}، فرهنگ نشر نو، چاپ اول، 1396، ص. 30.}
	
	حق بر آینده سبز، در پی بررسی و تبیین حقوقی است که در پاسخ به چالش‌های ایجاد شده توسط تغییرات‌اقلیمی در آینده بوده و تعهدات بین‌نسلی برای نسل حاضر به عنوان نسل مسئول ایجاد می‌نماید. در این میان باید پرسید که سازکارها و نهاد‌های بین‌المللی که با هدف مقابله با تغییرات‌اقلیمی شکل گرفته‌اند، در اتخاذ رویکرد‌های خود تا چه اندازه به کیفیت زندگی شهروندان آینده کره زمین توجه نموده‌اند؟ آیا قواعدی وجود دارد که در عین توجه به نیاز‌های نسل حاضر، تمهیداتی برای نسل‌های آینده اندیشیده باشد؟ اگر وجود دارد، این قواعد از چه اصولی برمی‌خیزند و چگونه نسل حاضر را متعهد می‌نمایند؟


اثبات تعهد بین‌المللی دولت در حفظ و حمایت از محیط‌زیست، از دو دریچه قابل توجه است. یکی حق بر محیط‌زیست سالم به عنوان یکی از مصادیق حقوق بشر، که در این صورت مسئولیت بین‌المللی دولت در نقض حقوق انسانی اتباع خود مطرح می‌گردد؛ دیگری از دریچه محیط‌زیست، به عنوان میراث مشترک بشریت.\footnote{افشاری، مریم، «تاثیرات تغییرات آب و هوایی بر صلح و امنیت انسانی»، پایان‌نامه دکتری دانشگاه شهید بهشتی، 1389، ص. 230.}

به لحاظ روشی نگارنده این اثر را زیر مجموعه حقوق محیط‌زیست دانسته و تبیین و نقض حق بر آینده سبز را نیز در قالب حفاظت از حقوق محیط‌زیست به عنوان امانت مشترک بشریت مورد توجه قرار داده است. البته تردیدی نیست که حقوق محیط‌زیست مبتنی بر علایق و منافع مشترک بشریت در حقوق شناخته شده برای افراد و نظریه حقوق بشر است، لذا تضادی میان این دو در طرح و بررسی مسئله وجود ندارد. چرا که رعایت جهانی حقوق و آزادی‌های بنیادین افراد، جزئی از علایق و منافع مشترک بشریت است.\footnote{امیر‌ارجمند، اردشیر، «حفاظت از محیط‌زیست و همبستگی بین‌المللی»، مجله تحقیقات حقوقی، شماره 15، دانشکده حقوق دانشگاه شهید بهشتی، 1373-1374، ص. 344.}


	\item اهمیت و ضرورت تحقیق
	
	
	
	اهمیت و ضرورت این تحقیق از چند بعد قابل توجه و بررسی می‌باشد.
	
	تغییرات قلب تپنده آینده هستند. به لحاظ ساختاری ظهور دورِ انسان و ورود کره زمین به دورِ ناپایداری و تغییر، ضرورت تحول ساخت‌ها و مبنا‌هایی که در دورِ پایداری و در پاسخ به ضرورت‌های آن دورِ ایجاد شده‌اند را، ایجاد نموده و تبیین اصول و مفاهیم جدید برای پاسخ به شرایط جدید را واجب نموده‌اند. 
	
	از نظر رویکردی، حقوق‌دانان بین‌المللی باید با توجه به آثار بلند‌مدت تغییرات‌اقلیمی، رویکرد‌های آینده‌نگرانه داشته و با پیش‌بینی‌ شرایط و آثار اقدامات امروزی در آینده، گام بردارد.
	 رویکرد آینده‌پژوهی با ارائه مدل‌های مختلف، و با محور قرار دادن درک ما از واقعیت‌های آینده، ضرورت شکل‌گیری مفاهیم جدید در این عرصه را مد نظر قرار می‌دهد.
	 بدیهی است اتخاذ چنین رویکردی نیازمند شناخت چالش‌های پیش‌روی جوامع است. 
	
	با مداقه در روند متحولانه تغییرات‌اقلیمی و در ابعاد جامع‌تر، دورِانسان در حقوق بین‌الملل و شاخه‌های آن، و انتقادات وارده به جامعه جهانی به جهت ناتوانی در ایجاد همبستگی برای مقابله با آثار فزاینده تغییرات‌اقلیمی و مهار انتشار گاز‌های گلخانه‌ای، ضرورت توجه به حق بر آینده سبز برای کودکان و نسل‌های آینده تبیین می‌شود. 
	
	
	
	
	%بحران‌های محیط‌زیستی همچون انقراض گونه‌ها و تغییرات‌اقلیمی شرایط زیستی کره زمین را چنان تحت تاثیر قرار دادند که تا چندین صد سال آثار و تبعات آن بر پیکره زمین وجود خواهد داشت و حیات را متاثر خواهد نمود. این امر ضرورت  تبیین رابطه بین‌نسلی و حقوق نسل‌های آینده را آشکار نموده و به قولی بعد زمانی قاعده حقوقی را به آینده گسترش داده است. 
	
	%دولت‌ها وسازمان‌های بین‌المللی، دیگر تابعان منحصر به فرد حقوق بین‌الملل نیستند. نه تنها ستم دیدگان و قربانیان دولت‌های ستمگر، بلکه تمامی انسان‌هایی که با رنج هر موجود انسانی همدردی کنند، می‌توانند از این دولت‌ها بخواهند که حد خود را نگه دارند. 
	
	
	
	
	
	
	
	\item پرسش‌های تحقیق
	\begin{enumerate}
		
		\item پرسش‌های اصلی
		
		رابطه نسل حاضر با نسل‌های آینده در دور انسان چگونه خواهد بود؟ این پرسش محوری‌ترین پرسش متن حاضر خواهد بود. اما نیل بدین پرسش با پیش‌فرض‌های همراه است که ضرورت دارد مورد بررسی قرار بگیرد. بنا بر گزارش‌های علمی، دامنه آثار تغییرات‌اقلیمی به قدری گسترده است که ذهن نسل حاضر را در مورد زندگی فرزندان و نسل‌های آینده درگیر می‌نماید. در این زمانه که تحت عنوان دورِانسان مورد شناسایی واقع شده باید توجه داشت که ویژگی عمده طبیعت دیگر ثبات نبوده و نمی‌توان مبتنی بر آن حرکت نمود. اگر ویژگی زمانه و دورِ انسان تغییر در معنای منفی آن باشد باید پرسید که آنها در چه شرایطی زندگی خواهند نمود و رابطه نسل‌ حاضر با آنها چیست؟ 
		
		متن حاضر به صورت خاص به تغییرات‌اقلیمی به عنوان یکی از بزرگ‌ترین نشانه‌ها و ویژگی‌های دورِانسان پرداخته و در پی بررسی آثار این تغییرات بر حق بر آینده سبز می‌باشد. همچنین در پی بررسی وضعیت حقوق بین‌الملل محیط‌زیست در پاسخ به چالش‌های تغییرات‌اقلیمی بوده و سازکار‌ها و  آرمان‌های حمایت از حق بر آینده سبز در برابر این تغییرات‌ را مورد بررسی قرار خواهد داد.
			
			
	
		
		\item پرسش‌های فرعی\\
					اگر در پرسش‌ اصلی متن حاضر دقیق شویم به سوالات متعددی خواهیم رسید که تحت عنوان پرسش‌های فرعی باید به آنها پاسخ دهیم. این پرسش‌ها حول محور دو مفهوم تغییرات‌اقلیمی و حق بر آینده سبز شکل می‌گیرند و در پی معنا و تعریف آن هستند. جایگاه این تغییرات در نظام هستی با توجه به ویژگی‌های آن چیست؟ ذینفعان حق بر آینده سبز چه گروهی هستند؟ و تغییرات‌اقلیمی چگونه بر حقوق آنها تاثیر می‌گذارد؟ 
						
همچنین پاسخ‌های نظام حقوق بین‌المللی  محیط‌زیست به تغییرات‌اقلیمی و آثار آن بر حق بر آینده سبز مورد پرسش می‌باشد. اصول حقوق تغییرات‌اقلیمی چیست؟ و چه آرمان‌هایی برای حمایت از حق بر آینده سبز در برابر تغییرات‌اقلیمی در این نظام تعریف شده است؟
			

	
		
	\end{enumerate}
	
	\item فرضیات تحقیق
	
	\begin{enumerate}
		
		\item فرضیه اصلی\\
		پس از تفَوّق انسان بر طبیعت در دورِانسان نظم و تعادل پیشین طبیعت تغییر کرده و وارد عرصه جدیدی شده است.   حقوق تغییرات‌اقلیمی در همگام‌سازی رابطه اجتماعی با نظم جدید طبیعت، علی‌رغم تعریف مفاهیم و اصول گسترده در این حوزه ناتوان بوده و نتوانسته شدت آثار تغییرات‌اقلیمی  را تقلیل دهد، به نحوی که روند گرمایش جهانی حکایت از آثار بلند‌مدت گرمایش جهانی و تاثیرپذیری آیندگان از عملکرد نسل‌های پیشین دارد.
		
		\item فرضیه رقیب \\
		حقوق تغییرات‌اقلیمی با ایجاد اجماع بین‌المللی در خصوص گرمایش جهانی کشور‌ها را در دو طیف توسعه یافته و در حال توسعه به یکدیگر نزدیک نموده و با اسناد، تصمیمات و راهبرد‌های مختلف گاز‌های گلخانه‌ای را تا حد زیادی کاهش داده است. همچنین با توجه به اصول توسعه پایدار و عدالت و انصاف بین‌نسلی حیات اجتماعی کنونی را در آینده تامین نموده است.
		
		
	\end{enumerate}


	
	
	\item پیشینه تحقیق

		اصطلاح دورِانسان از جمله اصطلاحات جدید در حوزه حقوق بین‌الملل محیط‌زیست می‌باشد، که در ایران بسیار محدود مورد استفاده بوده است. 	اشاره به دورِانسان و رابطه مبنایی آن برای حق بر آینده سبز از جمله موارد جدید در پژوهش‌های فارسی می‌باشد. 
		
		در میان تحقیقات حقوق داخلی، اگر‌چه بحث تغییرات‌اقلیمی و آثار آن در منابع مختلفی مورد اشاره و بحث قرار گرفته است، اما تاکنون تاثیر آن بر آینده و حق بر آینده سبز مورد اشاره واقع نشده است. 
		این تحقیقات عمدتا به تاثیر تغییرات‌اقلیمی بر صلح و امنیت پرداخته‌اند. 

بررسی تاثیر تغییرات‌اقلیمی بر حقوق کودکان و بر نسل‌های آینده از جمله مطالعات جدید در این حوزه بوده که تاکنون مورد بررسی واقع نشده است. همچنین در خصوص رابطه زمانی که تغییرات‌اقلیمی میان نسل حاضر و نسل‌های آینده ایجاد می‌کند و آرمان‌هایی که از حق بر آینده سبز حمایت می‌کند، که شامل میراث مشترک بشریت، خیر مشترک بشریت، نگرانی مشترک بشریت و عدالت و انصاف بین‌نسلی است، تاکنون تخقیقی صورت نگرفته است. اگرچه که این مفاهیم به صورت جداگانه  تبیین شده و به تفصیل مورد بحث قرار گرفته‌اند. 
		
		همچنین از جمله نوآوری‌های این متن تبیین حقوق نسل‌های آینده در دستگاه عدالت جان رالز بوده است. 
		
		
	
در پژوهش‌های خارجی، مقاله و اسناد مختلفی در خصوص تاثیر تغییرات‌اقلیمی بر حقوق کودکان وجود دارد که به اعتقاد نگارنده تمام زوایای حقوق کودک را مورد توجه قرار نداده‌اند. این متن مطالعات صورت گرفته در حوزه رابطه تفییرات‌اقلیمی با حقوق کودکان را به صورت جامع و یکجا گردآوری نموده است. 

در تبیین حقوق نسل‌های آینده کتب و مقالات بسیاری وجود دارد که به نحو مقتضی از آنها بهره برده شده است، تحلیل این حقوق در دستگاه عدالت جان رالز، در پژوهش‌های بین‌المللی مشاهده نشده و از جمله نوآوری‌های متن می‌باشد. 
		
		


	
	\item رویکرد و روش تحقیق\\
تفکر در مورد آینده در عصر تغییرات یک ضرورت است. بدون داشتن رویکرد آینده‌پژوهی به آینده می‌توان واکنش نشان داد، اما کنش نمی‌توان نشان داد. در متن حاضر سعی گردیده رویکرد آینده‌پژوهی حقوقی مورد توجه قرار گیرد و بدین وسیله با توجه به تغییرات‌اقلیمی اختلافات و مشکلات آینده تا حد ممکن بررسی و پاسخ‌های حقوقی موجود مورد بررسی قرار بگیرد. همچنین در بخش‌هایی از بحث به سناریونویسی به عنوان یکی از شیوه‌های آینده‌پژوهی حقوقی (پیشگیری)، با تاکید بر داده‌های علمی روز پرداخته‌ایم.
	
گردآوری مطالب عمدتا به صورت کتابخانه‌ای و در فضای مجازی با تکیه بر کتاب‌ها و مقالات روز است. همچنین در پژوهش و نوشته حاضر،  مکانیسم‌ها و سازکار‌های اجرایی   در کنوانسیون‌ها  مورد توجه قرار گرفته و اشاره شده است. روش اعمال شده در این نوشتار، روشی توصیفی-تحلیلی است که طی آن ضمن تبیین عِلی روابط و نسبت بین پدیده‌ها، با اتخاذ رهیافت ایده‌آلیستی، به تاکید بر اصول و ارزش‌های جامعه بین‌المللی و جایگاه تعیین کننده آنها در نظم و نسق‌دهی به دیدگاه‌های عاملان روابط بین‌الملل در حوزه حقوق تغییرات اقلیمی پرداخته‌ شده است. 

	 
	
	نگارنده در نگارش متن پایان‌نامه سعی نموده است که به جهت نهادینه ساختن اصول و مفاهیم علمی و حقوقی در نظام حقوقی کشورمان، تا حد ممکن از استفاده از اصطلاحات و واژگان غیر‌فارسی پرهیز و از برابر‌های مناسب فارسی بهره ببرد. از جمله می‌توان به استفاده از برابر بوم‌سازگان به جای اکوسیستم اشاره نمود. 
	
		
	\item سازماندهی مطالب\\
	در فصل اول، به بیان زمینه‌ها و چالش‌های بوجود آمده پرداخته و مفاهیم پایه در این حوزه تعریف و تبیین می‌شود. در این فصل، ابتدا به آغاز تغییرات می‌پردازیم و دورِانسان را تعریف می‌نماییم. سپس به تعریف تغییرات‌اقلیمی به عنوان یکی از برجسته‌ترین تغییراتِ دورِانسان پرداخته و آثار آن بر نظام اقلیم و جوامع انسانی بررسی می‌گردد. و در پایان، حق بر آینده سبز و ذینفعان آن معرفی می‌گردند. در این فصل، جهت‌دهی و تثبیت علمی چالش صورت می‌پذیرد، به نحوی که مبنا مباحث بعدی در دو فصل بعدی باشد. 
	
	در فصل دوم، حق بر آینده سبز به تفصیل مورد بحث و بررسی قرار می‌گیرد. در این فصل جنبه‌ها و حقوق مختلف کودک با توجه به کنوانسیون حقوق کودک به عنوان سند مرجع و مادر، و دیگر اسناد موجود مورد بررسی قرار گرفته و سعی گردیده با توجه به آثار تغییرات‌اقلیمی در بازه زمان، تکالیف دولت‌ها در برابر کودکان مشخص گردد. همچنین حقوق نسل‌های آینده مورد بررسی قرار گرفته است. جایگاه نسل‌های آینده در حقوق بین‌الملل مورد بررسی قرار گرفته و رابطه آنها با تغییرا‌اقلیمی معین گشته است. 
	
	
	در فصل سوم، به  سازکار‌های بین‌المللی مقابله با تغییرات‌اقلیمی پرداخته‌ایم. این سازکار‌ها همچون هیات بین‌دولتی تغییرات‌اقلیمی، کنوانسیون چارچوبی تغییرات‌اقلیمی و... که در جهت حمایت از مشترکات بین‌المللی (اقلیم) شکل گرفته‌اند و ناظم رابطه اجتماعی با طبیعت هستند، می‌پردازیم و در مبحث بعدی با توجه به آرمان‌های حمایت از حق بر آینده سبز، مبانی و اصول حاکم بر نهاد‌ها را مورد توجه قرار می‌دهیم. 
	
	
\end{enumerate}






































%حقوق محیط‌زیست با تمام اصول، قواعد، نهاد‌ها و مکانیزم‌هایش از رابطه اجتماعی با طبیعت سخن می‌گوید. این حقوق بواسطه درهم شکستن مرز‌های سیاسی میان دولت‌ها و صحبت از خیر مشترک جامعه جهانی، در تحول و عبور از حقوق بین‌الملل سنتی و رسیدن به حقوق بین‌الملل معاصر نقشی موثر داشته است. ایجاد مفاهیم خیر مشترک، میراث مشترک و نگرانی مشترک بشریت و توجه به ارزش‌های بنیادین جامعه بین‌المللی در مقابل اصل حاکمیت دولت‌ها، اصل تقابل و اصل اداره خودپسندانه منافع دولتی، بر ضرورت آینده‌نگری در فرایند‌های قاعده‌گذاری برای پاسخ به چالش‌های جامعه بین‌المللی تاکید نموده و  حمایت از حق بر آینده سبز را تبیین می‌نماید. 












 




%عدم‌پایداری و تغییر، از جمله ویژگی‌های بسیار برجسته و عمده دورِانسان است. طبیعت، بواسطه مداخله انسان، در ابعاد وسیعی در حال تغییر و واکنش نسبت به کنش‌های انسانی است. تغییرات‌اقلیمی از جمله این واکنش‌ها بوده 
%که با افزایش دما و ایجاد نقصان در چرخه‌های طبیعی نهادینه شده در ادوار قبلی زمین، 
%که حیات روابط و تمدن انسانی را به صورت خاص و حیات تمام موجودات را به صورت عام در معرض خطر نابودی قرار داده است.


% نظم چرخه‌های زیستی در ابتد با مداخله انسان دچار کاستی و نقصان شد و سپس به واسطه اهمیت بنیادین و مبنایی طبیعت برای زیست و حیات موجودات، نظم جوامع انسانی نیز دچار اختلال گردید.

%به لحاظ زمانی چنین اختلالی در حال و زمان آینده رخ می‌هد. به عیارت دیگر آثار رفتار و عملکرد انسان امروزی، در دایره زمانی بزرگتری از عمر او، و بر رابطه اجتماعی شهروندان آینده زمین یعنی کودکان و نسل‌های آینده اثر می‌گذارد. 



 % این موضوعات ما را به پرسش‌های مختلفی رهنمون می‌سازد از جمله اینکه تغییرات اقلیم چیست و چه آثاری بر کودکان و نسل‌های آینده دارد.به همین منظور در فصل دوم، به بررسی حقوق کودکان و نسل‌های آینده و رابطه  آن با تغییرات‌اقلیمی می‌پردازیم.  همچنین با توجه به ساختار حقوق بین‌الملل، جامعه بین‌المللی چه پاسخی برای تحدید بلند مدت تغییرات‌اقلیمی اندیشیده و این راهکار‌ها تا چه اندازه به حق بر آینده سبز توجه داشته‌اند. 
 
 
 
 
 
 







%محیط زیست مبنا و پایه ای اساسی برای نیل به حقوق بشر بوده و کیفیت آن تضمین کننده کرامت ذاتی بشر و دیگر موجودات می باشد.  این پایان نامه از دو رویکرد استفاده نموده است به این صورت که در بحث حقوق کودکان به رویکردی وجودی و انسان گرا گرایش پیدا نموده است و در بحث حقوق نسل های آینده اساسا به دلیل عدم وجود آنها، ملاک را ارزش محیط زیست گرفته و با توجه به تخریبی که نسل حاضر بر پیکره محیط زیست و به خصوص اقلیم وارد می آورد به تبیین تعهدات نسل حاضر در برابر نسل های آینده می پردازد. بدیهی است که با توجه به عمر طولانی اقلیم و گونه های زیستی  که بین 10 هزار سال تا 1 میلیون سال است در برابر عمر حداکثر 100 ساله انسان ،  امکان اصالت دهی به وجود انسان و تعریف و حمایت از حقوق محیط زیست زیر مجموعه حقوق بشر ممکن نیست. باید به محیط زیست با توجه به کیفیت و کارآیی در بازه زمانی خودش توجه نمود و سپس از تعهدات نسل فاعل در برابر نسلهای مفعول در دایره عمر اقلیم و محیط زیست سخن راند. 

%در این حالت نسل حاضر در نقش فاعل، مجری و امانت دار باید به گونه ای سکان کشتی حیات را اداره نماید که کشتی حیات، نسل های حاضر و آینده را به سمت و سوی خیر مشترک و عدالت و کرامت ذاتی اانسان و دیگر موجودات با یکدیگر و توامان هم و در قالب اصول توسعه پایدار، مسئولیت مشترک اما متفاوت و ... به پیش رود. در این مسیر از جمله موانع خودخواهی و ملیت گرایی حاکمیت هاست که بدنبال منفعت عمومی خود از خیر مشترک جامعه جهانی که با همبستگی بدست می آید، می گذرند. 

%رویکرد متن حاضر به کودکان و نسل‌های آینده یکسان نمی‌تواند باشد. اسناد حقوق بشری مختلف از جمله کنوانسیون حقوق کودک با رویکرد حقوق مشارکتی به کودکان توجه داشته و آنها را دارنده حقوق می‌شمارد. در حالی که در خصوص حقوق نسل‌های آینده اساسا سند الزام‌آوری وجود نداشته و بواسطه ناموجود بودن آنها صحبت از حقوق مشارکتی بی معناست. اما با توجه به اصول حقوقی و ارزش‌های بشری، و با توجه به بحرانی که تغییرات‌اقلیمی برای نسل‌های آینده بوجود می‌آورد، حمایت از آنها ضروری می‌نماید. 









